Methoden van Jacobi en Gauss-Seidel convergeren enkel indien de matrix A van het stelsel \textit{diagonaal dominant is}.\\
Dus: een element op de diagonaal moet in absolute waarde groter zijn dan de som van de absolute waarden van alle andere elementen die zich op \textit{dezelfde rij} bevinden. Dit is voldoende maar niet altijd \textit{nodig} voor convergentie.\\\\
Dus:\\
$|\alpha+1| > 1$ en\\
$1 > |\alpha|$\\\\
Dit geldt voor:
$\alpha \in ]0,1[$\\

Dit is echter een voldoende voorwaarde en geen nodige voorwaarde.
Een nodige en voldoende voorwaarde is dat de spectrale radius
(grootste eigenwaarde volgens absolute waarde) van de matrix G kleiner is dan 1.\\
De matrix G is gedefinieerd als: $D^{-1}$ + L + U, waar D een matrix is die de diagonaal van de matrix A bevat, L de onderdriehoeksmatrix is en U de bovendriehoeksmatrix is. (voor meer informatie zie boek bij jacobi, daar is de G matrix gedefinieerd).
